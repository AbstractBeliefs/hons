\documentclass[a4paper]{article}

\usepackage{lastpage}
\usepackage{fancyhdr}
\pagestyle{fancy}

\addtolength{\oddsidemargin}{-.875in}
\addtolength{\evensidemargin}{-.875in}
\addtolength{\headwidth}{1.75in}
\addtolength{\textwidth}{1.75in}

\lhead{Gareth Pulham}
\rhead{40099603}
\lfoot{MEng Software Engineering}
\cfoot{\today}
\rfoot{\thepage\ of \pageref{LastPage}}

\begin{document}
    \begin{titlepage}
        \title{SOC10101 Initial Project Overview}
        \author{Gareth Pulham, 40099603}
        \date{\today}
        \maketitle
        \thispagestyle{empty}
    \end{titlepage}

    \section{Title of Project}
        GPU Accelerated Calculation of Radio Viewsheds

    \section{Project Content and Milestones}
        This project intends to research the validity and effectiveness of GPU acceleration of radio viewshed
        calculations. The project aims to take common algorithms that are used to complete these calculations and
        convert them from purely CPU to GPU-accelerated versions. These resulting algorithms can then be compared with
        their CPU counterparts and their relative performance examined. Following this, the converted algorithms will
        be packaged in a reusable manner for others to benefit from.
    \section{Deliverables}
        \begin{itemize}
            \item CPU and GPU viewshed calculation tools used to generate results and performance data
            \item Documentation listing the findings of comparing the resultant data.
            \item A report into the validity of the GPU accelerated viewshed calculations, and possible application of
                the findings to other areas
            \item Summary poster of work
        \end{itemize}
    \section{Target Audience for the Deliverables}
        The target audience for the deliverables will mainly exist in two areas: firstly, those who need to generate
        viewsheds, who presently need to either purchase proprietary tools to do so in a fast or automated manner, and
        secondly those who are interested in GPU accelerated GIS (Geographic Information Systems) applications, as the
        field has a large number of parallelisable problems, many of which are at present unaccelerated.
    \section{The Work to be Undertaken}
        \begin{itemize}
            \item Initial investigation into the current state of the art for accelerated GIS and ray casting
            \item Investigation into approaches to GPU accelerate GIS calculations
            \item Implementation of CPU based viewshed calculations
            \item Implementation of GPU based viewshed calculations
            \item Recording performance of the above calculations
            \item Writing a report documenting all of the above
        \end{itemize}
    \section{Additional Information / Knowledge Required}
        \begin{itemize}
            \item An in depth understanding of GPU programming
            \item The strengths and weaknesses of CUDA vs OpenCL
            \item An understanding of current pathfinding and ray tracing algorithms to apply
            \item An understanding of GIS data formats
        \end{itemize}
    \section{Information Sources that Provide a Context for the Project}
        Wikipedia contains some information on viewsheds and viewshed analysis from a broad point of view, mentioning
        specifically some instances where it can be used:
        \begin{itemize}
            \item https://en.wikipedia.org/wiki/Viewshed
            \item https://en.wikipedia.org/wiki/Viewshed\_analysis
        \end{itemize}

        Some prior art exists in terms of viewshed calculation, which is commonly included in GIS tools such as ArcGIS,
        but also in online hiking tools, though these sources tend to be proprietary, or limited in scope or resolution
        , or slow:
        \begin{itemize}
            \item http://pro.arcgis.com/en/pro-app/tool-reference/3d-analyst/viewshed.htm
            \item http://www.heywhatsthat.com/
        \end{itemize}

        Additionally, GPUs have already seen some use in GIS applications, which often utilise large datasets of 2D/3D
        floating point fields. Nvidia has some case studies showing the use of their Tesla compute cards:
        \begin{itemize}
            \item http://www.nvidia.com/object/geographic-information-systems.html
        \end{itemize}
    \section{The Importance of the Project}
        Viewsheds are surprisingly commonly used, though often not by that name. It's not uncommon for people to want
        to know what can be seen from, say, local high spots like Arthur's Seat, and recently many people in Edinburgh
        have been able to see gas flares at Exxon's Mossmorran NGL plant in Fife. Viewsheds can help plan the
        construction of infrastructure like this, as well as other high buildings that may have visibility
        (specifically, not-visible) requirements as part of their planning applications in protected areas.

        Aside from these casual uses, fast, free GIS tools such as viewshed calculators are in demand for projects like
        Tegola/HUBS ( http://www.tegola.org.uk/ ), who presently have to use slow and low resolution tools, or even use
        maps and in person investigations. Better tools can lead directly to the faster delivery of radio enabled
        broadband to remote communities.
    \section{The Key Challenges to be Overcome}
        The biggest challenge will be to learn the paradigms and practices used in GPU programming, as it's a new field
        to me. After this, it will be important to carefully test the new implementations to ensure that any speedup or
        results are not due to individual machine specifications or quirks.
\end{document}
